\documentclass[a4paper,12pt]{article}
\usepackage{tabularx}
\usepackage{booktabs}


\usepackage{xepersian} 
\settextfont{XB Yas.ttf} 

\usepackage{geometry}
\geometry{top=3cm,bottom=3cm,left=2.5cm,right=2.5cm}

\begin{document}

\begin{titlepage}
    \begin{center}
        \vspace*{2cm}
        
        {\Huge \textbf{مستند نیازمندی‌های سیستم و بیزینس}}\\[1cm]
        
        {\LARGE درس مهندسی نرم افزار}\\[0.5cm]
        {\LARGE آبان / سال ۱۴۰۴}\\[2cm]
        
        {\huge \textbf{عنوان سیستم: ROOSTERME}}\\[2cm]
        
        {\Large اعضای تیم:}\\[0.5cm]
        {\Large مریم رجبی \quad ۴۰۱۴۶۳۱۳۷}\\
        {\Large مریم پورحسن \quad ۴۰۱۴۶۳۱۲۷}\\[3cm]
        
        \vfill
    \end{center}
\end{titlepage}

% فهرست مطالب
\tableofcontents
\newpage

% -----------------------------
\section{سند نیازمندی‌های سیستم (SRS)}
\vspace{0.5cm}

\subsection{مقدمه}
\vspace{0.3cm}

\subsubsection{هدف سند}
\vspace{0.2cm}

هدف این سند، ارائه یک توضیح دقیق، رسمی و شفاف از \textbf{نیازمندی‌های سیستم اپلیکیشن RoosterMe} است. این اپلیکیشن یک سامانه ترکیبی است که از \textbf{آلارم و چالش‌های ذهنی و فکری} بهره می‌برد تا به کاربر کمک کند در \textbf{زمان مشخص و مقرر} بیدار شود.

\vspace{0.3cm}

این سند به صورت مشخص تعیین می‌کند که سیستم:
\begin{enumerate}
    \item چه وظایفی را باید انجام دهد،
    \item چگونه باید رفتار کند، و
    \item تحت چه محدودیت‌ها و شرایطی باید توسعه یابد.
\end{enumerate}

\vspace{0.3cm}

علاوه بر این، این سند \textbf{پیوند مستقیم بین نیازمندی‌های بیزینسی (BRD) و رفتار واقعی سیستم} برقرار می‌کند، تا اطمینان حاصل شود که تمامی نیازمندی‌های تجاری به صورت صحیح و دقیق در پیاده‌سازی سیستم رعایت شده‌اند و هیچ ناهماهنگی بین هدف بیزینسی و عملکرد واقعی سیستم وجود ندارد.

\vspace{0.3cm}

به بیان ساده‌تر، این سند یک \textbf{راهنمای رسمی و جامع} است که توسعه‌دهندگان، طراحان و سایر ذینفعان پروژه را قادر می‌سازد تا \textbf{سیستم RoosterMe را به درستی و مطابق با نیازهای کاربران و اهداف تجاری طراحی و پیاده‌سازی کنند}.

\vspace{0.5cm}

\subsubsection{دامنه سیستم}
\vspace{0.2cm}

سیستم \textbf{RoosterMe} یک \textbf{اپلیکیشن موبایل} است که به کاربران کمک می‌کند تا در \textbf{زمان مقرر و به شیوه‌ای هوشمند} از خواب بیدار شوند. این سیستم از \textbf{آلارم‌های هوشمند} و \textbf{چالش‌های بیدارباش} مانند حل مسائل ریاضی، تمرین‌های تمرکزی و بازی‌های کوتاه ذهنی بهره می‌برد تا کاربر مجبور شود فعالانه در فرآیند بیدار شدن شرکت کند و به راحتی به خواب بازنگردد.

\vspace{0.3cm}

علاوه بر وظیفه اصلی بیدارباش، سیستم امکانات جانبی زیر را ارائه می‌دهد:
\begin{enumerate}
    \item \textbf{ثبت Streak:} ردیابی تعداد روزهای متوالی که کاربر موفق به بیدار شدن شده است.
    \item \textbf{ایجاد گروه دوستان:} امکان اشتراک‌گذاری پیشرفت و چالش‌ها با دوستان و خانواده.
    \item \textbf{رتبه‌بندی (Leaderboard):} نمایش مقایسه‌ای عملکرد کاربران در قالب رتبه‌بندی، برای افزایش انگیزه.
    \item \textbf{ذخیره‌سازی ابری:} نگهداری داده‌ها و پیشرفت کاربران به صورت امن در فضای ابری، تا اطلاعات حتی با تغییر دستگاه حفظ شوند.
    \item \textbf{اعلان‌های یادآوری:} ارسال یادآوری‌ها و نوتیفیکیشن‌ها برای کمک به کاربر در پیگیری اهداف و زمان بیدار شدن.
\end{enumerate}

\vspace{0.3cm}

دامنه این سند شامل \textbf{تمام قابلیت‌های مورد نیاز برای نسخه اولیه سیستم} است و هدف آن ارائه یک \textbf{راهنمای شفاف برای توسعه سیستم} می‌باشد. این سند شامل \textbf{معماری سیستم، طراحی داخلی یا جزئیات پیاده‌سازی} نمی‌شود و تمرکز آن صرفاً بر \textbf{نیازمندی‌های کارکردی و عملکردی اپلیکیشن} است.

\vspace{0.3cm}

به عبارت دیگر، دامنه مشخص می‌کند که \textbf{سیستم چه کاری باید انجام دهد} و چه امکاناتی باید در نسخه اولیه فراهم باشد، بدون آنکه وارد جزئیات نحوه طراحی یا ساخت شود.


\subsubsection{مخاطبان سند}
\vspace{0.3cm}

این سند برای گروه‌های مختلف ذینفعان و نقش‌های کلیدی در پروژه تهیه شده است تا هر گروه بتواند \textbf{وظایف و نیازمندی‌های سیستم RoosterMe} را به درستی درک کند. مخاطبان اصلی سند شامل موارد زیر هستند:

\vspace{0.2cm}

\begin{itemize}
    \item \textbf{تیم توسعه:} شامل برنامه‌نویسان و مهندسان نرم‌افزار که مسئول پیاده‌سازی و توسعه سیستم هستند.
    \item \textbf{تحلیلگران سیستم:} افرادی که نیازمندی‌ها و جریان کاری سیستم را بررسی و تحلیل می‌کنند.
    \item \textbf{مدیر پروژه:} شخصی که بر کل فرآیند توسعه، زمان‌بندی و تخصیص منابع نظارت دارد.
    \item \textbf{ذی‌نفعان بیزینسی:} افرادی که اهداف تجاری و نیازمندی‌های کسب‌وکار را تعیین و پیگیری می‌کنند.
    \item \textbf{تسترها:} افرادی که مسئول طراحی و اجرای تست‌ها برای اطمینان از صحت عملکرد سیستم و استخراج \textbf{Test Case} هستند.
\end{itemize}

\vspace{0.3cm}

استفاده از این سند توسط مخاطبان ذکر شده، اطمینان می‌دهد که \textbf{تمام نیازمندی‌های سیستم به شکل یکپارچه و دقیق درک و پیاده‌سازی شوند} و هماهنگی بین تیم توسعه و اهداف بیزینسی برقرار باشد.


\subsubsection{تعاریف و اصطلاحات}
\vspace{0.3cm}

در این بخش، \textbf{تعاریف و اصطلاحات کلیدی} مورد استفاده در سند ارائه می‌شود تا \textbf{مفاهیم به صورت شفاف و یکسان برای همه ذینفعان} قابل درک باشد:

\vspace{0.2cm}

\begin{itemize}
    \item \textbf{آلارم چالشی:} آلارمی که برای خاموش شدن نیازمند انجام یک \textbf{چالش ذهنی یا حرکتی} است. این مکانیزم باعث می‌شود کاربر فعالانه در فرآیند بیدار شدن شرکت کند.
    \item \textbf{Challenge:} فعالیتی مانند حل مسئله یا بازی کوتاه ذهنی که کاربر باید انجام دهد تا آلارم خاموش شود.
    \item \textbf{Streak:} تعداد روزهای متوالی که کاربر موفق شده به‌موقع بیدار شود و این داده برای انگیزه‌بخشی و نمایش پیشرفت کاربر استفاده می‌شود.
    \item \textbf{Leaderboard:} لیست رتبه‌بندی کاربران بر اساس عملکرد آن‌ها در چالش‌ها و بیدارباش، برای ایجاد رقابت و انگیزه.
    \item \textbf{Synced Data:} داده‌هایی که بین دستگاه کاربر و سرور به صورت \textbf{همگام‌شده} ذخیره و به‌روزرسانی می‌شوند تا اطلاعات همیشه به‌روز و هماهنگ باشد.
\end{itemize}

\vspace{0.3cm}

این تعاریف به توسعه‌دهندگان، تحلیلگران، تسترها و سایر ذینفعان کمک می‌کند تا \textbf{همه مفاهیم یکسان فهمیده شوند} و از هرگونه ابهام یا سوءتفاهم جلوگیری شود.


\subsubsection{مراجع و مستندات مرتبط}
\vspace{0.3cm}

در تدوین این سند، منابع و مستندات زیر مورد استفاده قرار گرفته‌اند تا \textbf{استانداردها و بهترین شیوه‌ها} در طراحی و توسعه سیستم رعایت شود:

\vspace{0.2cm}

\begin{itemize}
    \item \textbf{IEEE 830 – Software Requirements Specification Standard:} استاندارد بین‌المللی برای نوشتن سند نیازمندی‌های نرم‌افزار، که روش‌های دقیق برای تعریف نیازمندی‌ها ارائه می‌دهد.
    \item \textbf{Business Requirements Document (BRD) – RoosterMe (2025):} سند نیازمندی‌های بیزینسی مرتبط با پروژه RoosterMe که اهداف تجاری و انتظارات ذینفعان را مشخص می‌کند.
    \item \textbf{راهنمای طراحی UI/UX اپلیکیشن‌های موبایل (Google Material Design):} استانداردها و دستورالعمل‌های طراحی رابط کاربری برای اپلیکیشن‌های موبایل، برای اطمینان از تجربه کاربری بهینه.
    \item \textbf{اصول MoSCoW برای اولویت‌بندی نیازمندی‌ها:} روش اولویت‌بندی نیازمندی‌ها با دسته‌بندی به Must have، Should have، Could have و Won't have، برای مدیریت بهتر توسعه و زمان‌بندی پروژه.
\end{itemize}

\vspace{0.3cm}

این منابع باعث می‌شوند که سند \textbf{مطابق استانداردهای بین‌المللی و نیازمندی‌های بیزینسی} تدوین شود و تمامی تصمیمات طراحی و توسعه بر اساس مستندات معتبر و قابل استناد باشند.


\subsection{شرح کلی سیستم}
\vspace{0.5cm}

\subsubsection{چشم‌انداز کلی}
\vspace{0.3cm}

سیستم \textbf{RoosterMe} یک \textbf{اپلیکیشن موبایل هوشمند} است که هدف اصلی آن کمک به کاربران برای \textbf{بیدار شدن در زمان مقرر} است. این اپلیکیشن از ترکیبی از \textbf{آلارم‌های هوشمند} و \textbf{چالش‌های ذهنی و حرکتی} بهره می‌برد تا کاربر به شکل فعالانه در فرآیند بیدار شدن شرکت کند و از خواب بازگشت سریع جلوگیری شود.

\vspace{0.3cm}

علاوه بر عملکرد اصلی بیدارباش، سیستم دارای \textbf{قابلیت‌های اجتماعی و انگیزشی} است، که شامل موارد زیر می‌شود:
\begin{itemize}
    \item \textbf{رقابت و رتبه‌بندی (Leaderboard):} امکان مشاهده عملکرد خود نسبت به دیگر کاربران و ایجاد انگیزه برای بهبود عملکرد.
    \item \textbf{ثبت Streak:} پیگیری تعداد روزهای متوالی موفقیت در بیدار شدن به موقع، برای ایجاد انگیزه و تشویق کاربر.
    \item \textbf{امکانات گروهی:} ایجاد گروه دوستان و اشتراک‌گذاری پیشرفت‌ها و چالش‌ها، که تعامل اجتماعی و حس مسئولیت‌پذیری را افزایش می‌دهد.
\end{itemize}

\vspace{0.3cm}

به طور کلی، چشم‌انداز سیستم این است که یک تجربه \textbf{کاربر محور، تعاملی و انگیزشی} ایجاد کند تا کاربر نه تنها به موقع بیدار شود، بلکه انگیزه داشته باشد این رفتار را به شکل مستمر ادامه دهد و از مزایای اجتماعی و شخصی آن بهره‌مند شود.


\subsubsection{نقش‌ها و کاربران سیستم}
\vspace{0.3cm}

سیستم \textbf{RoosterMe} دارای چند نقش اصلی است که هر یک وظایف و دسترسی‌های مشخصی دارند. شرح نقش‌ها به شرح زیر است:

\vspace{0.2cm}

\begin{itemize}
    \item \textbf{کاربر عادی:}  
    کاربران عادی می‌توانند \textbf{آلارم‌های شخصی خود را ایجاد و تنظیم کنند}، \textbf{چالش‌های ذهنی و حرکتی} را انجام دهند و پس از بیدار شدن، \textbf{گزارش‌های عملکرد و پیشرفت خود} را مشاهده کنند. این نقش برای استفاده روزمره و تعامل مستقیم با سیستم طراحی شده است.

    \vspace{0.2cm}

    \item \textbf{کاربر گروه:}  
    کاربران گروه می‌توانند علاوه بر عملکردهای کاربر عادی، \textbf{با سایر اعضای گروه تعامل داشته باشند}، پیشرفت و نتایج خود را با دیگران به اشتراک بگذارند و \textbf{رتبه و جایگاه خود در گروه یا Leaderboard} را مشاهده کنند. این نقش برای افزایش انگیزه و رقابت اجتماعی طراحی شده است.

    \vspace{0.2cm}

    \item \textbf{مدیر سیستم (ادمین):}  
    ادمین مسئول \textbf{مدیریت داده‌های سرور، کنترل دسترسی کاربران، مدیریت نسخه‌های سیستم و پشتیبانی فنی} است. این نقش همچنین می‌تواند در نسخه‌های بعدی، تنظیمات پیشرفته و \textbf{نظارت بر سلامت سیستم و هماهنگی با تیم توسعه} را انجام دهد. این نقش برای حفظ امنیت، عملکرد صحیح و پایداری سیستم حیاتی است.
\end{itemize}

\vspace{0.3cm}

این دسته‌بندی نقش‌ها باعث می‌شود هر کاربر \textbf{با دسترسی مناسب و وظایف مشخص} بتواند با سیستم تعامل کند و مدیریت پروژه و داده‌ها به صورت منظم انجام شود.


\subsubsection{محیط عملیات}
\vspace{0.3cm}

سیستم \textbf{RoosterMe} در محیط‌های مشخصی طراحی شده است تا عملکرد صحیح و بهینه داشته باشد. محیط عملیات شامل موارد زیر است:

\vspace{0.2cm}

\begin{itemize}
    \item \textbf{سیستم‌عامل:}  
    اپلیکیشن روی دستگاه‌های موبایل با سیستم‌عامل‌های \textbf{Android} و \textbf{iOS} قابل اجرا است. برای استفاده از امکاناتی مانند \textbf{ذخیره‌سازی ابری، گروه‌های دوستان و اعلان‌ها}، اتصال به اینترنت ضروری است. این موضوع تضمین می‌کند که داده‌ها همواره \textbf{به‌روز و همگام‌سازی شده} باقی بمانند.

    \vspace{0.2cm}

    \item \textbf{سخت‌افزار:}  
    برای استفاده کامل از سیستم، کاربر به یک \textbf{گوشی هوشمند} نیاز دارد که قابلیت \textbf{نمایش اعلان‌ها و پخش صدا} را داشته باشد. این ویژگی‌ها برای عملکرد درست آلارم‌ها و تجربه کاربری مطلوب ضروری هستند.
\end{itemize}

\vspace{0.3cm}

با توجه به این محیط، کاربران اطمینان خواهند داشت که سیستم \textbf{به درستی و بدون مشکل} در دستگاه‌های پشتیبانی‌شده اجرا می‌شود و امکانات اجتماعی، انگیزشی و چالشی به شکل کامل در دسترس است.



\subsubsection{اجزای اصلی سیستم}
\vspace{0.5cm}

سیستم \textbf{RoosterMe} از چند ماژول اصلی تشکیل شده است که هر یک مسئول بخش مشخصی از عملکرد اپلیکیشن هستند. این ماژول‌ها عبارتند از:

\vspace{0.2cm}

\begin{itemize}
    \item \textbf{ماژول آلارم هوشمند:}  
    مسئول ایجاد، مدیریت و فعال‌سازی \textbf{آلارم‌های هوشمند} است. این ماژول زمان‌بندی دقیق، تنظیم صدا و نوع هشدار و همچنین تعامل با ماژول چالش‌ها برای خاموش کردن آلارم را کنترل می‌کند.

    \vspace{0.2cm}

    \item \textbf{ماژول چالش‌ها:}  
    مدیریت \textbf{چالش‌های ذهنی و حرکتی} که کاربر باید انجام دهد تا آلارم خاموش شود. این ماژول شامل انواع فعالیت‌ها، درجه سختی، و پیگیری موفقیت کاربران در هر چالش است.

    \vspace{0.2cm}

    \item \textbf{ماژول حساب کاربری:}  
    مدیریت اطلاعات کاربر، ثبت‌نام، ورود، و ویرایش پروفایل. این ماژول همچنین ارتباط کاربران با گروه‌ها و ویژگی‌های اجتماعی را تسهیل می‌کند.

    \vspace{0.2cm}

    \item \textbf{ماژول Streak و تحلیل عملکرد:}  
    پیگیری تعداد روزهای متوالی موفقیت کاربران (\textbf{Streak}) و ارائه گزارش‌ها و تحلیل‌های عملکرد برای ایجاد انگیزه و بهبود رفتار بیدار شدن.

    \vspace{0.2cm}

    \item \textbf{ماژول اجتماعی (گروه‌ها و رتبه‌بندی):}  
    امکان ایجاد گروه دوستان، اشتراک‌گذاری پیشرفت‌ها، مشاهده عملکرد دیگران و نمایش رتبه کاربران (\textbf{Leaderboard}) برای افزایش انگیزه و تعامل اجتماعی.

    \vspace{0.2cm}

    \item \textbf{ماژول اعلان‌ها:}  
    ارسال \textbf{اعلان‌ها و یادآوری‌ها} به کاربران برای پیگیری آلارم‌ها، چالش‌ها و پیشرفت روزانه، به منظور تضمین تعامل مستمر با سیستم.

    \vspace{0.2cm}

    \item \textbf{ماژول ذخیره‌سازی ابری و Sync:}  
    نگهداری داده‌های کاربران به صورت امن در فضای ابری و همگام‌سازی (\textbf{Sync}) اطلاعات بین دستگاه‌ها، برای اطمینان از دسترسی به داده‌ها و پیشرفت‌های کاربر در هر دستگاه.
\end{itemize}

\vspace{0.3cm}

این ماژول‌ها با همکاری یکدیگر یک تجربه \textbf{یکپارچه، هوشمند و انگیزشی} ایجاد می‌کنند و اطمینان می‌دهند که سیستم به شکل کامل و قابل اعتماد عمل کند.


\section{نیازمندی‌های عملکردی (FunctionalRequirements)}
\vspace{0.5cm}

بخش \textbf{نیازمندی‌های عملکردی (FunctionalRequirements)} شامل تمام رفتارها و قابلیت‌هایی است که سیستم \textbf{RoosterMe} باید ارائه دهد تا اهداف کاربر و نیازهای تجاری برآورده شود. این نیازمندی‌ها مشخص می‌کنند که سیستم چه کاری باید انجام دهد، چگونه باید با کاربران و داده‌ها تعامل کند و چه خروجی‌هایی باید تولید کند.

\vspace{0.3cm}

ویژگی‌های مهم نیازمندی‌های عملکردی عبارتند از:

\begin{enumerate}
    \item \textbf{توصیف دقیق رفتار سیستم:}  
    هر نیازمندی عملکردی باید به طور روشن مشخص کند که سیستم در موقعیت خاص چه کاری انجام می‌دهد و نتیجه مورد انتظار چیست.
    
    \item \textbf{شفافیت و قابل پیگیری بودن:}  
    نیازمندی‌ها باید به گونه‌ای تعریف شوند که \textbf{توسعه‌دهندگان، طراحان و تسترها} بتوانند به راحتی آن‌ها را پیاده‌سازی و آزمون کنند.
    
    \item \textbf{ارتباط با نیازمندی‌های بیزینسی (BRD):}  
    هر نیازمندی عملکردی باید با هدف بیزینسی مرتبط باشد تا اطمینان حاصل شود که توسعه سیستم دقیقاً مطابق اهداف تجاری پیش می‌رود.
    
    \item \textbf{تعیین اولویت:}  
    با استفاده از روش \textbf{MoSCoW}، نیازمندی‌ها به دسته‌های Must have، Should have، Could have و Won't have تقسیم می‌شوند تا تیم توسعه بداند کدام ویژگی‌ها حیاتی هستند و کدام می‌توانند در نسخه‌های بعدی اضافه شوند.
    
\end{enumerate}


\subsection{قالب استاندارد IEEE برای تعریف نیازمندی‌های عملکردی}
\vspace{0.3cm}

برای نظم و یکپارچگی، هر نیازمندی عملکردی مطابق قالب استاندارد \textbf{IEEE 830} تعریف می‌شود. این قالب شامل بخش‌های زیر است:

\vspace{0.2cm}

\begin{itemize}
    \item \textbf{ID:} FR-XX-YY (شماره شناسایی یکتا برای هر نیازمندی)
    \item \textbf{عنوان:} یک عبارت کوتاه و گویا برای نیازمندی
    \item \textbf{شرح:} توضیح کامل و دقیق از عملکرد مورد انتظار
    \item \textbf{پیش‌شرط:} شرایط یا وضعیت‌هایی که باید قبل از اجرای عملکرد موجود باشند
    \item \textbf{پس‌شرط:} وضعیت یا نتیجه‌ای که بعد از اجرای عملکرد باید حاصل شود
    \item \textbf{ورودی‌ها:} داده‌ها، اطلاعات یا رویدادهایی که سیستم دریافت می‌کند
    \item \textbf{خروجی‌ها:} نتایج، اعلان‌ها یا تغییراتی که سیستم تولید می‌کند
    \item \textbf{معیار پذیرش:} شرایطی که نشان می‌دهد نیازمندی با موفقیت پیاده‌سازی شده است
    \item \textbf{اولویت (MoSCoW):} تعیین اهمیت نیازمندی برای نسخه اولیه و نسخه‌های بعدی
    \item \textbf{ارجاع به BRD:} پیوند با نیازمندی بیزینسی مربوطه
\end{itemize}

\vspace{0.3cm}

استفاده از این قالب باعث می‌شود که هر نیازمندی \textbf{قابل فهم، قابل ردیابی و قابل آزمون} باشد و توسعه‌دهندگان و تسترها بتوانند به راحتی آن را بررسی و پیاده‌سازی کنند. همچنین این ساختار تضمین می‌کند که نیازمندی‌ها از نظر \textbf{پیش‌نیازها، ورودی‌ها، خروجی‌ها و معیارهای پذیرش} کاملاً روشن و بدون ابهام باشند.

به عبارت دیگر، بخش نیازمندی‌های عملکردی یک \textbf{نقشه دقیق و عملیاتی} از سیستم ارائه می‌دهد و پل بین \textbf{اهداف تجاری و پیاده‌سازی فنی} است.


\subsection{فهرست نیازمندی‌های عملکردی سیستم}

در این زیر‌بخش، فهرست کامل نیازمندی‌های عملکردی سیستم ارائه می‌شود. هر نیازمندی مطابق قالب استاندارد IEEE ۸۳۰ تدوین شده است تا تمامی اجزای اصلی شامل \lr{ID}، عنوان، شرح، ورودی‌ها، خروجی‌ها، پیش‌شرط، پس‌شرط، معیار پذیرش و اولویت به صورت دقیق و قابل ردیابی مشخص باشد.

این ساختار به تیم توسعه، تست و مدیریت پروژه کمک می‌کند تا نیازمندی‌ها را بدون ابهام درک کرده، آن‌ها را به صورت دقیق پیاده‌سازی کنند و در مراحل تست، به سادگی امکان اعتبارسنجی آن وجود داشته باشد.

نیازمندی‌های عملکردی سیستم \textbf{RoosterMe} بر اساس ماژول‌های اصلی زیر دسته‌بندی شده‌اند:

\begin{itemize}
    \item ماژول آلارم (\lr{Alarm Module})
    \item ماژول چالش‌ها (\lr{Challenge Module})
    \item ماژول حساب کاربری (\lr{User Account Module})
    \item ماژول Streak و تحلیل عملکرد (\lr{Streak \& Analytics Module})
    \item ماژول اجتماعی شامل گروه‌ها و رتبه‌بندی (\lr{Social Module})
    \item ماژول اعلان‌ها (\lr{Notification Module})
    \item ماژول ذخیره‌سازی ابری و همگام‌سازی (\lr{Cloud Sync Module})
\end{itemize}

برای هر ماژول، مجموعه نیازمندی‌ها به صورت مجزا ارائه می‌شود تا ساختار سیستم شفاف، قابل مدیریت و قابل توسعه باقی بماند. در ادامه، نیازمندی‌های عملکردی مرتبط با هر ماژول، به تفصیل فهرست شده‌اند.

\vspace{0.5cm}

\subsubsection*{\textbf{ماژول آلارم}}
\vspace{0.5cm}

\textbf{:ID} FR-AL-۰۱

\textbf{عنوان:} ایجاد آلارم

\textbf{شرح:} کاربر باید بتواند یک آلارم جدید در سیستم ایجاد کند. هنگام ایجاد آلارم، کاربر می‌تواند زمان، روزهای تکرار، صدای آلارم و نوع چالش مورد نظر را تعیین کند. سیستم موظف است ورودی‌های کاربر را بررسی کرده و آلارم را به لیست آلارم‌ها افزوده و فعال نماید.  

\textbf{پیش‌شرط:} کاربر باید وارد سیستم شده باشد و دسترسی به اعلان‌ها (\lr{Notification Permission}) فعال باشد.  

\textbf{پس‌شرط:} آلارم جدید به لیست آلارم‌ها افزوده شده و در حافظه پایدار (Local Storage یا Sync) ذخیره می‌شود.  

\textbf{ورودی‌ها:} زمان آلارم، روزهای تکرار، صدای آلارم و نوع چالش (مثلاً ریاضی، اسکن QR، تکان دادن گوشی و …).  

\textbf{خروجی‌ها:} آلارم جدید در لیست آلارم‌ها نمایش داده شده و فعال می‌شود.  

\textbf{معیار پذیرش:} آلارم باید با تمام پارامترهای تعیین‌شده ذخیره شود، پس از ذخیره در لیست نمایش داده شود، در زمان مقرر فعال گردد و چالش انتخاب شده هنگام هشدار اجرا شود.  

\textbf{اولویت (MoSCoW):} Must

\textbf{ارجاع به :BRD} BR-۰۱


\vspace{0.5cm}
\vspace{0.5cm}


\textbf{:ID} FR-AL-۰۲

\textbf{عنوان:} ویرایش آلارم

\textbf{شرح:} کاربر باید بتواند پارامترهای یک آلارم موجود را تغییر دهد. این تغییرات می‌تواند شامل زمان، روزهای تکرار، صدای آلارم و نوع چالش باشد. سیستم موظف است تغییرات کاربر را ذخیره کرده و آلارم را با تنظیمات جدید فعال نگه دارد.

\textbf{پیش‌شرط:} آلارم مورد نظر باید در لیست آلارم‌ها موجود باشد.  

\textbf{پس‌شرط:} تغییرات اعمال شده ذخیره شده و آلارم با پارامترهای جدید فعال شود.  

\textbf{ورودی‌ها:} پارامترهای جدید آلارم (زمان، روزهای تکرار، صدا، نوع چالش).  

\textbf{خروجی‌ها:} پیام موفقیت در اعمال تغییرات و به‌روزرسانی آلارم در لیست نمایش داده شود.  

\textbf{معیار پذیرش:} تغییرات باید در لیست آلارم‌ها قابل مشاهده باشند و آلارم با پارامترهای جدید به درستی فعال شود.  

\textbf{اولویت (MoSCoW):} Should  

\textbf{ارجاع به :BRD} BR-۰۱

\vspace{0.5cm}
\vspace{0.5cm}
\vspace{0.5cm}
\vspace{0.5cm}


\textbf{:ID} FR-AL-۰۳

\textbf{عنوان:} حذف آلارم

\textbf{شرح:} کاربر باید بتواند یک آلارم موجود را به طور کامل حذف کند. سیستم موظف است پس از تأیید حذف، آلارم را از لیست آلارم‌ها پاک کرده و دیگر آن را فعال نکند.

\textbf{پیش‌شرط:} آلارم مورد نظر باید در لیست آلارم‌ها موجود باشد.

\textbf{پس‌شرط:} آلارم حذف شده و از لیست آلارم‌ها پاک شود.

\textbf{ورودی‌ها:} شناسه آلارم (Alarm ID) که کاربر قصد حذف آن را دارد.

\textbf{خروجی‌ها:} لیست آلارم‌ها پس از حذف و پیام موفقیت‌آمیز بودن عملیات.

\textbf{معیار پذیرش:} پس از حذف، آلارم دیگر در لیست نمایش داده نشود و در زمان تعیین شده اجرا نشود.

\textbf{اولویت (MoSCoW):} Must

\textbf{ارجاع به :BRD} BR-۰۱

\vspace{0.5cm}  % فاصله قبل از نیازمندی بعدی

\vspace{0.5cm}
\vspace{0.5cm}

\textbf{:ID} FR-AL-۰۴

\textbf{عنوان:} فعال شدن آلارم

\textbf{شرح:} آلارم باید در زمان تعیین‌شده به صدا درآید و چالش مربوط به آلارم برای کاربر نمایش داده شود تا کاربر بتواند آن را انجام دهد. سیستم موظف است اطمینان حاصل کند که آلارم دقیقاً در زمان برنامه‌ریزی شده اجرا شده و چالش با موفقیت شروع می‌شود.

\textbf{پیش‌شرط:} آلارم باید فعال باشد.

\textbf{پس‌شرط:} چالش مربوط به آلارم آغاز شده و آماده تعامل با کاربر باشد.

\textbf{ورودی‌ها:} زمان جاری سیستم (System Time) برای بررسی زمان اجرای آلارم.

\textbf{خروجی‌ها:} صدای آلارم فعال می‌شود و چالش روی صفحه نمایش داده می‌شود.

\textbf{معیار پذیرش:} آلارم باید دقیقاً در زمان تعیین شده اجرا شود و چالش به درستی نمایش داده شود.

\textbf{اولویت (MoSCoW):} Must

\textbf{ارجاع به :BRD} BR-۰۲


\vspace{0.5cm}
\vspace{0.5cm}

\subsubsection*{\textbf{ماژول چالش‌ها}}
\vspace{0.5cm}

\textbf{:ID} FR-CH-۰۱

\textbf{عنوان:} چالش ریاضی

\textbf{شرح:} سیستم باید یک مسئله ریاضی تولید کند و پاسخ کاربر را ارزیابی نماید. کاربر باید پاسخ صحیح را وارد کند تا آلارم خاموش شود. سیستم موظف است پاسخ کاربر را بررسی کرده و پیام صحیح یا غلط را نمایش دهد.

\textbf{پیش‌شرط:} آلارم مرتبط باید فعال شده باشد.

\textbf{پس‌شرط:} در صورت پاسخ صحیح، آلارم خاموش می‌شود و کاربر می‌تواند آلارم بعدی یا فعالیت دیگر را دنبال کند.

\textbf{ورودی‌ها:} پاسخ کاربر به مسئله ریاضی.

\textbf{خروجی‌ها:} پیام اعلام صحیح یا غلط بودن پاسخ و وضعیت آلارم (خاموش یا فعال).

\textbf{معیار پذیرش:} تنها در صورت پاسخ صحیح، آلارم خاموش شود و پیام مناسب نمایش داده شود.

\textbf{اولویت (MoSCoW):} Must

\textbf{ارجاع به :BRD} BR-۰۲,BR-۰۳

\vspace{0.5cm} 
\vspace{0.5cm}


\textbf{:ID} FR-CH-۰۲

\textbf{عنوان:} چالش حافظه

\textbf{شرح:} سیستم باید یک الگو (مثلاً رنگ‌ها، شکل‌ها یا دنباله‌ای از اعداد) نمایش دهد و کاربر باید آن را بازسازی کند. سیستم پاسخ کاربر را بررسی کرده و نتیجه را ثبت نماید.

\textbf{پیش‌شرط:} آلارم مرتبط باید در حال اجرا باشد.

\textbf{پس‌شرط:} نتیجه پاسخ کاربر ثبت شده و برای تحلیل عملکرد یا امتیازدهی آماده باشد.

\textbf{ورودی‌ها:} انتخاب‌ها یا بازسازی کاربر از الگو.

\textbf{خروجی‌ها:} پیام اعلام درست یا غلط بودن بازسازی و ثبت نتیجه.

\textbf{معیار پذیرش:} سیستم باید پاسخ کاربر را دقیق بررسی کند و پیام صحیح/غلط را نمایش دهد و نتیجه را ثبت کند.

\textbf{اولویت (MoSCoW):} Should

\textbf{ارجاع به :BRD} BR-۰۲,BR-۰۳


\vspace{0.5cm} 
\vspace{0.5cm}

\textbf{:ID} FR-CH-۰۳

\textbf{عنوان:} مدیریت سختی چالش

\textbf{شرح:} سیستم باید شدت یا سطح سختی چالش‌ها را بر اساس انتخاب کاربر یا تنظیمات پیش‌فرض تعیین کند. چالش‌های تولید شده باید با سطح سختی انتخاب‌شده هماهنگ باشند و تجربه مناسبی برای کاربر فراهم کنند.

\textbf{پیش‌شرط:} سطح سختی چالش توسط کاربر انتخاب شده باشد یا تنظیم پیش‌فرض تعیین شده باشد.

\textbf{پس‌شرط:} چالش تولید شده با سطح سختی انتخابی تطابق دارد و آماده اجرا برای کاربر است.

\textbf{ورودی‌ها:} سطح سختی انتخاب شده توسط کاربر (مثلاً آسان، متوسط، سخت).

\textbf{خروجی‌ها:} چالش مناسب با سطح سختی انتخاب‌شده تولید می‌شود.

\textbf{معیار پذیرش:} چالش تولید شده دقیقاً با سطح سختی انتخاب شده مطابقت داشته باشد و قابل حل باشد.

\textbf{اولویت (MoSCoW):} Could

\textbf{ارجاع به :BRD} BR-۰۳

\vspace{0.5cm} 
\vspace{0.5cm}

\subsubsection*{\textbf{ماژول کاربر}}

\vspace{0.5cm} 

\textbf{:ID} FR-US-۰۱

\textbf{عنوان:} ثبت‌نام

\textbf{شرح:} سیستم باید امکان ایجاد حساب کاربری جدید را برای کاربران فراهم کند. کاربر می‌تواند با وارد کردن ایمیل معتبر و رمز عبور، یک حساب کاربری ایجاد کند. سیستم صحت اطلاعات ورودی را بررسی کرده و در صورت موفقیت، حساب را ایجاد می‌نماید.

\textbf{پیش‌شرط:} اتصال اینترنت فعال باشد.

\textbf{پس‌شرط:} حساب کاربری جدید ایجاد شده و اطلاعات کاربر ذخیره شود.

\textbf{ورودی‌ها:} ایمیل و رمز عبور کاربر.

\textbf{خروجی‌ها:} پیام موفقیت‌آمیز بودن ثبت‌نام و فعال شدن حساب کاربری.

\textbf{معیار پذیرش:} پس از ثبت‌نام، کاربر قادر به ورود با ایمیل و رمز عبور انتخاب شده باشد.

\textbf{اولویت (MoSCoW):} Must

\textbf{ارجاع به :BRD} BR-۰۳

\vspace{0.5cm}  
\vspace{0.5cm}

\textbf{:ID} FR-US-۰۲

\textbf{عنوان:} ورود

\textbf{شرح:} سیستم باید امکان احراز هویت کاربران را فراهم کند. کاربر با وارد کردن ایمیل و رمز عبور می‌تواند وارد حساب خود شود و به داشبورد و امکانات اپلیکیشن دسترسی پیدا کند.

\textbf{پیش‌شرط:} حساب کاربری موجود باشد.

\textbf{پس‌شرط:} کاربر وارد سیستم شده و دسترسی به داشبورد و امکانات فراهم شود.

\textbf{ورودی‌ها:} ایمیل و رمز عبور کاربر.

\textbf{خروجی‌ها:} نمایش داشبورد و دسترسی به امکانات اپلیکیشن.

\textbf{معیار پذیرش:} پس از ورود، کاربر باید به داشبورد و امکانات اپلیکیشن دسترسی کامل داشته باشد.

\textbf{اولویت (MoSCoW):} Must

\textbf{ارجاع به :BRD} BR-۰۳

\vspace{0.5cm}
\vspace{0.5cm}

\subsubsection*{\textbf{ماژول Streak و گزارش}}
\vspace{0.5cm}


\textbf{:ID} FR-ST-۰۱

\textbf{عنوان:} ثبت Streak

\textbf{شرح:} سیستم باید موفقیت کاربر در بیدار شدن هر روز را ثبت کند. پس از انجام موفق چالش مرتبط با آلارم، سیستم شمارنده Streak را افزایش می‌دهد تا کاربر بتواند روند مداومت خود را مشاهده کند.

\textbf{پیش‌شرط:} کاربر چالش مرتبط با آلارم را با موفقیت انجام داده باشد.

\textbf{پس‌شرط:} شمارنده Streak به‌روز شده و مقدار جدید ذخیره شود.

\textbf{ورودی‌ها:} نتیجه موفقیت یا عدم موفقیت آلارم و چالش.

\textbf{خروجی‌ها:} مقدار عددی جدید Streak نمایش داده شود.

\textbf{معیار پذیرش:} پس از انجام موفق چالش، شمارنده Streak باید افزایش یابد و مقدار جدید به درستی نمایش داده شود.

\textbf{اولویت (MoSCoW):} Should

\textbf{ارجاع به :BRD} BR-۰۴

\vspace{0.5cm} 
\vspace{0.5cm}

\textbf{:ID} FR-ST-۰۲

\textbf{عنوان:} نمایش گزارش

\textbf{شرح:} سیستم باید امکان نمایش روند بیداری کاربر، درصد موفقیت و ساعات بیداری را ارائه دهد. اطلاعات عملکردی کاربر به صورت نمودارها و آمار در داشبورد نمایش داده می‌شود تا کاربر بتواند روند پیشرفت خود را مشاهده و تحلیل کند.

\textbf{پیش‌شرط:} داده‌های عملکردی مرتبط با Streak و چالش‌ها موجود باشد.

\textbf{پس‌شرط:} داشبورد گزارش‌ها به‌روز شده و قابل مشاهده باشد.

\textbf{ورودی‌ها:} داده‌های عملکرد کاربر (موفقیت‌ها، ساعات بیداری، تعداد Streak).

\textbf{خروجی‌ها:} نمودارها، آمار و اطلاعات روند بیداری در داشبورد.

\textbf{معیار پذیرش:} داده‌های نمایش داده شده باید دقیق و مطابق با اطلاعات عملکرد کاربر باشد و شامل درصد موفقیت و روند Streak باشد.

\textbf{اولویت (MoSCoW):} Could

\textbf{ارجاع به :BRD} BR-۰۴

\vspace{0.5cm}
\vspace{0.5cm}

\subsubsection*{\textbf{ماژول اجتماعی}}
\vspace{0.5cm}


\textbf{:ID} FR-SO-۰۱

\textbf{عنوان:} ایجاد گروه

\textbf{شرح:} سیستم باید امکان ایجاد گروه توسط کاربر را فراهم کند. کاربر می‌تواند نام گروه را وارد کرده و سیستم گروه جدیدی با شناسه منحصر به فرد ایجاد کند. این گروه قابل استفاده برای اشتراک‌گذاری چالش‌ها و مقایسه عملکرد با دوستان و اعضای دیگر است.

\textbf{پیش‌شرط:} کاربر وارد سیستم شده باشد.

\textbf{پس‌شرط:} گروه جدید ایجاد شده و شناسه گروه اختصاص داده شود.

\textbf{ورودی‌ها:} نام گروه وارد شده توسط کاربر.

\textbf{خروجی‌ها:} شناسه گروه جدید و پیام موفقیت ایجاد گروه.

\textbf{معیار پذیرش:} پس از ایجاد گروه، کاربر باید بتواند گروه را مشاهده کند و شناسه آن معتبر باشد.

\textbf{اولویت (MoSCoW):} Should

\textbf{ارجاع به :BRD} BR-۰۵

\vspace{0.5cm}
\vspace{0.5cm}

\textbf{:ID} FR-SO-۰۲

\textbf{عنوان:} عضویت در گروه

\textbf{شرح:} کاربر باید بتواند با وارد کردن کد گروه، به گروه اضافه شود. سیستم باید صحت کد گروه را بررسی کرده و عضویت کاربر را ثبت کند تا بتواند در فعالیت‌ها و چالش‌های گروه شرکت کند.

\textbf{پیش‌شرط:} گروه مورد نظر وجود داشته باشد.

\textbf{پس‌شرط:} عضویت کاربر در گروه ثبت شود.

\textbf{ورودی‌ها:} کد گروه وارد شده توسط کاربر.

\textbf{خروجی‌ها:} پیام موفقیت عضویت و به‌روزرسانی لیست اعضای گروه.

\textbf{معیار پذیرش:} کاربر باید بتواند پس از عضویت در گروه، به امکانات و فعالیت‌های گروه دسترسی داشته باشد.

\textbf{اولویت (MoSCoW):} Could

\textbf{ارجاع به :BRD} BR-۰۵

\vspace{0.5cm}
\vspace{0.5cm}
\subsubsection*{\textbf{ماژول رتبه‌بندی}}
\vspace{0.5cm}

\textbf{:ID} FR-LB-۰۱

\textbf{عنوان:} رتبه‌بندی کاربران

\textbf{شرح:} سیستم باید عملکرد کاربران را در گروه‌ها بررسی کرده و لیست رتبه‌بندی (Leaderboard) را نمایش دهد. این رتبه‌بندی می‌تواند شامل تعداد موفقیت‌ها، Streak و امتیازهای کسب شده باشد تا انگیزه کاربران افزایش یابد.

\textbf{پیش‌شرط:} گروه دارای اعضای فعال باشد و داده عملکرد کاربران موجود باشد.

\textbf{پس‌شرط:} Leaderboard به‌روز شده و قابل مشاهده است.

\textbf{ورودی‌ها:} داده‌های عملکرد کاربران در گروه.

\textbf{خروجی‌ها:} لیست مرتب شده کاربران بر اساس عملکرد و نمایش رتبه‌ها.

\textbf{معیار پذیرش:} رتبه‌بندی باید دقیق و مطابق با داده‌های عملکرد کاربران باشد و در داشبورد گروه نمایش داده شود.

\textbf{اولویت (MoSCoW):} Should

\textbf{ارجاع به :BRD} BR-۰۵

\vspace{0.5cm}
\vspace{0.5cm}

\subsubsection*{\textbf{ماژول اعلان و ذخیره‌سازی}}
\vspace{0.5cm}

\textbf{:ID} FR-NT-۰۱

\textbf{عنوان:} اعلان یادآوری خواب

\textbf{شرح:} سیستم باید امکان ارسال اعلان به کاربر را جهت یادآوری زمان خواب فراهم کند. این اعلان پیش از زمان خواب تعیین‌شده توسط کاربر فعال شده و به او هشدار می‌دهد که برای داشتن خواب کافی آماده شود.  

\textbf{پیش‌شرط:} مجوز اعلان (\lr{Notification Permission}) فعال باشد.

\textbf{پس‌شرط:} اعلان روی دستگاه نمایش داده شود.

\textbf{ورودی‌ها:} زمان تنظیم‌شده برای یادآوری خواب.

\textbf{خروجی‌ها:} پیام اعلان به کاربر.

\textbf{معیار پذیرش:} اعلان باید دقیقاً در زمان مقرر نمایش داده شود و در صورت غیرفعال بودن اعلان، به کاربر هشدار مناسب داده شود.

\textbf{اولویت (MoSCoW):} Could

\textbf{ارجاع به :BRD} BR-۰۳,BR-۰۴,BR-۰۲

\vspace{0.5cm}
\vspace{0.5cm}

\textbf{:ID} FR-CL-۰۱

\textbf{عنوان:} ذخیره‌سازی ابری داده‌ها

\textbf{شرح:} سیستم باید تمام آلارم‌ها، تنظیمات، داده‌های Streak، عملکرد چالش‌ها و اطلاعات کاربر را با سرور همگام‌سازی کند. این همگام‌سازی باید در پس‌زمینه انجام شده و تضمین کند که داده‌ها در تمام دستگاه‌های کاربر یکسان باشند.  

\textbf{پیش‌شرط:} اتصال اینترنت فعال باشد.

\textbf{پس‌شرط:} داده‌ها در سمت سرور و دستگاه کاربر به‌روز و یکسان شوند.

\textbf{ورودی‌ها:} داده‌های کاربر (آلارم‌ها، تنظیمات، گزارش‌ها، Streak و ...).

\textbf{خروجی‌ها:} پیام یا وضعیت تأیید همگام‌سازی موفق.

\textbf{معیار پذیرش:} داده‌ها باید بدون خطا روی سرور ذخیره شوند و در صورت وجود داده جدید، Sync باید به‌صورت کامل انجام شود.

\textbf{اولویت (MoSCoW):} Should

\textbf{ارجاع به :BRD} BRD-۰۶

\vspace{0.5cm}
\vspace{0.5cm}

\subsection{مدیریت تغییرات و تاریخچه نسخه‌ها}
\vspace{0.5cm}

«مدیریت تغییرات» در اسناد مهندسی نرم‌افزار فرآیندی است که طی آن هرگونه اصلاح، 
به‌روزرسانی، اضافه‌شدن یا حذف‌شدن محتوا در سند SRS قابل‌ردیابی و قابل‌پیگیری می‌شود. 
هدف اصلی این بخش ایجاد یک مکانیزم شفاف برای نگهداری سابقه نسخه‌هاست تا تمام افراد 
تیم توسعه، تحلیل‌گران، طراحان، مدیر پروژه و سایر ذی‌نفعان دقیقاً بدانند:

\begin{itemize}
    \item چه تغییری در سند ایجاد شده است؟
    \item چرا این تغییر انجام شده است؟
    \item چه شخصی مسئول ایجاد این تغییر بوده است؟
    \item این تغییر در چه تاریخی اعمال شده است؟
    \item نسخه فعلی سند با نسخه‌های قبلی چه تفاوتی دارد؟
\end{itemize}

وجود چنین سیستمی باعث می‌شود از هرگونه سردرگمی، دوباره‌کاری یا سوءتفاهم میان اعضای 
تیم جلوگیری شود و همه بدانند کدام نسخه معتبرترین و آخرین نسخه است.

\subsubsection*{اهمیت این بخش}

مدیریت تغییرات به دلایل زیر حیاتی است:

\begin{enumerate}
    \item \textbf{افزایش شفافیت:} تمام تاریخچه تغییرات در یک مکان ثبت می‌شود و هر کسی می‌تواند روند تکامل سند را مشاهده کند.
    \item \textbf{ردیابی تصمیمات:} اگر تصمیمی در گذشته گرفته شده باشد، با بررسی تاریخچه نسخه‌ها می‌توان علت و زمان آن را پیدا کرد.
    \item \textbf{جلوگیری از تعارض نسخه‌ها:} در پروژه‌های تیمی ممکن است هر فرد نسخه‌ای از سند داشته باشد. وجود بخش مدیریت تغییرات باعث می‌شود همه بدانند کدام نسخه رسمی است.
    \item \textbf{مسئولیت‌پذیری:} با ثبت نام فرد یا تیم تغییر دهنده، مشخص می‌شود چه کسی مسئول کدام قسمت است.
    \item \textbf{کنترل کیفیت:} نسخه‌سازی درست باعث می‌شود بتوان کیفیت سند و تغییرات آن را ارزیابی کرد.
\end{enumerate}

در نتیجه، بخش «مدیریت تغییرات و تاریخچه نسخه‌ها» یکی از حیاتی‌ترین قسمت‌های یک سند SRS 
به‌شمار می‌آید، زیرا تضمین می‌کند سند در طول زمان به‌صورت کنترل‌شده تکامل پیدا کند، 
همه تغییرات مستند باشند و تیم پروژه بتواند با اطمینان از اسناد معتبر و به‌روز 
استفاده نماید.

\noindent\textbf{برای مدیریت تغییرات، از یک Revision Log یا Revision History استفاده می‌شود که در آن اطلاعات ذخیره می‌گردد.}
\vspace{0.5cm}
\vspace{0.5cm}

\subsection{پیوست‌ها}
\vspace{0.5cm}
\vspace{0.5cm}

\subsubsection{واژه‌نامه اصطلاحات}

این بخش شامل توضیح دقیق و شفاف اصطلاحات کلیدی و تخصصی مورد استفاده در سند \lr{SRS} است. هدف از ارائه واژه‌نامه، یکسان‌سازی درک تمامی اعضای تیم از مفاهیم فنی و بیزینسی، جلوگیری از سوء‌برداشت‌ها و افزایش وضوح و یکپارچگی سند می‌باشد. وجود این بخش به توسعه‌دهندگان، تحلیل‌گران، طراحان و سایر ذی‌نفعان کمک می‌کند تا هنگام مطالعه سند با مفاهیم مهم و کاربردی کاملاً آشنا شوند و ابهامی در ارتباط با اصطلاحات تخصصی پیش نیاید.
\vspace{0.5cm}

\textbf{آلارم چالشی:}
نوعی آلارم است که برای خاموش شدن نیازمند انجام یک فعالیت ذهنی یا حرکتی توسط کاربر است. این فعالیت می‌تواند شامل حل مسائل ریاضی، بازسازی یک الگو، اسکن کد \lr{QR}، تکان دادن گوشی یا انجام حرکات فیزیکی مشخص باشد. هدف از این طراحی، جلوگیری از خاموش کردن سریع آلارم و اطمینان از بیداری کامل کاربر است. آلارم چالشی یک جزء اصلی مکانیزم بیدارباش اپلیکیشن است و تضمین می‌کند کاربر با انجام یک فعالیت فعالانه از خواب بیدار شود.
\vspace{0.5cm}

\textbf{Challenge (چالش):}
فعالیتی تعاملی که هنگام فعال شدن آلارم به کاربر نمایش داده می‌شود و کاربر باید آن را با موفقیت انجام دهد تا آلارم خاموش گردد. چالش‌ها می‌توانند ذهنی یا حرکتی باشند و میزان سختی آن‌ها قابل تنظیم است. این بخش به کاربر انگیزه می‌دهد تا درگیر فرآیند بیدار شدن شود و از خواب دوباره جلوگیری می‌کند. نمونه‌هایی از چالش‌ها شامل حل مسائل ریاضی، بازی‌های حافظه‌ای، بازسازی الگو و اسکن کد \lr{QR} است.
\vspace{0.5cm}

\textbf{Streak (استریک):}
تعداد روزهای متوالی که کاربر موفق شده است در زمان مقرر بیدار شود و چالش‌های مرتبط را با موفقیت پشت سر بگذارد. این ویژگی انگیزه‌بخش است و با ایجاد حس پیشرفت و رقابت، کاربر را ترغیب می‌کند تا به صورت مداوم از اپلیکیشن استفاده کند. در صورتی که کاربر یک روز موفق نشود، مقدار استریک صفر شده و شمارش از ابتدا آغاز می‌شود. این مفهوم برای ایجاد حس مسئولیت‌پذیری و استمرار در استفاده از اپلیکیشن اهمیت دارد.
\vspace{0.5cm}

\textbf{Leaderboard (لیدربورد):}
سیستم رتبه‌بندی کاربران بر اساس عملکرد آن‌ها در اپلیکیشن و فعالیت‌های گروهی است. معیارهایی مانند درصد موفقیت در بیدارباش، استریک و انجام چالش‌ها برای تعیین رتبه کاربران استفاده می‌شود. لیدربورد جنبه رقابتی ایجاد کرده و باعث افزایش تعامل کاربران، انگیزه‌بخشی و حس پیشرفت می‌شود. کاربران می‌توانند با مشاهده موقعیت خود نسبت به دیگران اهداف جدید تعیین کنند.
\vspace{0.5cm}

\textbf{Synced Data (داده همگام‌شده):}
داده‌هایی هستند که علاوه بر ذخیره محلی در دستگاه، با سرور نیز همگام‌سازی شده‌اند. این داده‌ها شامل آلارم‌ها، تنظیمات کاربری، گزارش‌ها و سابقه عملکرد است. همگام‌سازی داده‌ها باعث می‌شود کاربران بتوانند اطلاعات خود را هنگام تعویض دستگاه یا استفاده از چند دستگاه به راحتی بازیابی کنند و هیچ اطلاعاتی از دست نرود. همچنین امنیت و پایداری داده‌ها در فضای ابری افزایش می‌یابد.
\vspace{0.5cm}

\textbf{MoSCoW (مسکو):}
یک روش استاندارد برای اولویت‌بندی نیازمندی‌ها است که آن‌ها را به چهار گروه تقسیم می‌کند:
\begin{itemize}
    \item \textbf{Must:} الزامی و ضروری؛ این نیازمندی‌ها باید حتماً در نسخه اولیه سیستم پیاده‌سازی شوند.
    \item \textbf{Should:} مهم و با اهمیت؛ در صورت امکان باید اجرا شوند اما عدم پیاده‌سازی آن‌ها باعث عملکرد غیرقابل قبول سیستم نمی‌شود.
    \item \textbf{Could:} مطلوب و در صورت امکان اجرا می‌شوند؛ کم‌اهمیت‌تر بوده و اولویت پایین دارند.
    \item \textbf{Won’t:} در این نسخه اجرا نمی‌شوند؛ ممکن است در نسخه‌های بعدی لحاظ شوند.
\end{itemize}
این روش کمک می‌کند تا تیم توسعه منابع را به بهترین شکل مدیریت کند و تمرکز خود را بر نیازمندی‌های حیاتی و اصلی بگذارد.
\vspace{0.5cm}

\textbf{Document Requirement Business یا :BRD}
سند نیازمندی‌های بیزینسی که اهداف، دامنه، ذی‌نفعان، محدودیت‌ها و نیازهای اصلی کسب‌وکار را مشخص می‌کند. سند \lr{SRS} بر اساس اطلاعات ارائه شده در \lr{BRD} تدوین می‌شود تا اطمینان حاصل شود که سیستم توسعه‌یافته با اهداف و انتظارات کسب‌وکار همسو است و تمامی نیازمندی‌های تجاری در طراحی و پیاده‌سازی رعایت می‌شوند.
\vspace{0.5cm}
\vspace{0.5cm}

\subsubsection{نمودارها}

در این بخش، تمام نمودارهای طراحی و تحلیل سیستم \textbf{RoosterMe} قرار گرفته‌اند. این نمودارها در پوشه \texttt{diagrams} موجود می‌باشند. هدف از ارائه این نمودارها ایجاد یک درک مشترک بین تیم توسعه، تحلیل‌گران و ذی‌نفعان است و به آن‌ها کمک می‌کند تا ساختار، عملکرد و جریان داده‌ها در سامانه را به صورت بصری و قابل فهم مشاهده کنند. استفاده از دیاگرام‌ها باعث کاهش سوء‌تفاهم، تسهیل ارتباطات بین تیم‌ها و ایجاد پایه‌ای محکم برای طراحی و پیاده‌سازی سیستم می‌شود.

\textbf{انواع دیاگرام‌ها و توضیح آن‌ها:}

\begin{itemize}
    \item \textbf{Diagram Use-case (نمودار مورد استفاده):}  
    این نمودار تعامل کاربران با سیستم و قابلیت‌های کلیدی آن را نشان می‌دهد. هر نقش کاربری مانند \textit{کاربر عادی}، \textit{کاربر گروه} یا \textit{مدیر سیستم} و وظایف آن‌ها در قالب use-case مشخص شده است. این نمودار کمک می‌کند تا محدوده عملکرد سیستم و نیازهای کاربران به صورت واضح و قابل پیگیری نمایش داده شود.
    
    \item \textbf{Diagram Activity (نمودار فعالیت‌ها):}  
    جریان فرآیندها، فعالیت‌ها و تصمیم‌گیری‌ها در سیستم را نمایش می‌دهد. برای مثال، مراحل فعال شدن آلارم، انجام چالش، ثبت Streak و ارسال اعلان‌ها به صورت گام‌به‌گام در این نمودار قابل مشاهده است. Activity Diagram دید جامعی از فرآیندهای پویا و تعاملات زمانی ارائه می‌دهد.
    
    \item \textbf{Diagram Sequence (نمودار توالی):}  
    ترتیب تعامل اجزاء سیستم، ماژول‌ها و APIها با یکدیگر را نشان می‌دهد. این نمودار نمایش می‌دهد که هر جزء سیستم چه زمانی پاسخ می‌دهد و چگونه پیام‌ها بین اجزاء تبادل می‌شوند. برای نمونه، تعامل ماژول آلارم با ماژول چالش یا ذخیره‌سازی ابری در زمان بیدارباش قابل ردیابی است.
    
    \item \textbf{Diagram Flow Data (نمودار جریان داده):}  
    مسیرهای جریان داده‌ها در سیستم و فرآیندهای پردازش آن‌ها را نمایش می‌دهد. DFD نشان می‌دهد که داده‌ها از کجا وارد سیستم می‌شوند، در کجا پردازش می‌شوند و به کجا منتقل می‌شوند. این نمودار برای تحلیل دقیق جریان اطلاعات، مدیریت داده و همگام‌سازی بین ماژول‌ها اهمیت دارد.
    
    \item \textbf{Diagram Class (نمودار کلاس‌ها):}  
    ساختار ایستا و کلاس‌های سیستم را به همراه روابط و ویژگی‌های آن‌ها نمایش می‌دهد. این نمودار نشان می‌دهد هر کلاس چه خصوصیات (\textit{Attributes}) و رفتارها (\textit{Methods}) دارد و چگونه با دیگر کلاس‌ها مرتبط است. Class Diagram پایه‌ای برای پیاده‌سازی شیء‌گرای سیستم و طراحی پایگاه داده است.
\end{itemize}

استفاده از این مجموعه نمودارها به تیم توسعه و ذی‌نفعان کمک می‌کند تا:
\begin{itemize}
    \item ساختار و عملکرد سیستم را بهتر درک کنند.
    \item تعامل کاربران با سامانه را بصورت تصویری مشاهده کنند.
    \item جریان داده‌ها و وابستگی‌ها بین ماژول‌ها را به‌صورت دقیق ردیابی کنند.
    \item پایه‌ای محکم برای طراحی نرم‌افزار و تصمیم‌گیری‌های توسعه ایجاد شود.
\end{itemize}
\vspace{0.5cm}
\textbf{نمودار ها در پوشه ی diagrams  اورده شده اند}
\vspace{0.5cm}
\vspace{0.5cm}

\subsubsection{فهرست منابع و استانداردها}

در تدوین این سند از منابع و استانداردهای زیر استفاده شده است تا صحت، اعتبار و کیفیت محتوای ارائه‌شده تضمین شود:

\begin{enumerate}
    \item \textbf{IEEE Std 830-1998} – IEEE Recommended Practice for Software Requirements Specifications: استاندارد بین‌المللی برای نگارش و ساختار سند نیازمندی‌های نرم‌افزار.
    
    \item \textbf{Business Requirements Document (BRD) – RoosterMe (2025)}: سند نیازمندی‌های بیزینسی پروژه RoosterMe که اهداف تجاری و الزامات کلیدی کاربران را مشخص می‌کند.
    
    \item \textbf{Google Material Design Guidelines} – راهنمای طراحی UI/UX اپلیکیشن‌های موبایل: برای ایجاد رابط کاربری کاربرپسند، استاندارد و سازگار با پلتفرم‌های موبایل.
    
    \item \textbf{MoSCoW Prioritization Method} – روش اولویت‌بندی نیازمندی‌ها: تعیین اهمیت و ضرورت پیاده‌سازی هر نیازمندی در نسخه‌های مختلف سیستم (Must/Should/Could/Won’t).
    
    \item \textbf{مقالات و منابع آموزشی مهندسی نرم‌افزار و مستندات داخلی تیم توسعه}: شامل کتاب‌ها، مقالات علمی و مستندات داخلی که برای تکمیل اطلاعات و استناد به روش‌های توسعه نرم‌افزار استفاده شده است.
\end{enumerate}
\vspace{0.5cm}
\vspace{0.5cm}

\subsubsection{Revision History}
\vspace{0.5cm}

\noindent
این بخش فهرستی از نسخه‌های سند، تاریخ انتشار هر نسخه، نویسنده و توضیح مختصر تغییرات اعمال شده ارائه می‌دهد. هدف از ارائه این جدول، ایجاد شفافیت، امکان ردیابی تغییرات و پیگیری دقیق تکامل سند \lr{SRS} است. با استفاده از این تاریخچه، تمام اعضای تیم توسعه، تحلیل‌گران، طراحان و سایر ذی‌نفعان می‌توانند به راحتی تفاوت نسخه‌ها را مشاهده کرده، مسئولیت تغییرات را مشخص کنند و از هرگونه سردرگمی یا دوباره‌کاری جلوگیری نمایند.
\vspace{0.5cm}

\begin{tabular}{|c|c|c|p{8cm}|}
\hline
\textbf{نسخه} & \textbf{تاریخ انتشار} & \textbf{نویسنده} & \textbf{تغییرات / توضیحات} \\
\hline
1 & 02.۰۸.۱۴۰۴ & مریم رجبی & ایجاد پیش‌نویس اولیه SRS و ساختار بخش‌ها \\
\hline
2 & 11.۰۸.۱۴۰۴ & مریم رجبی & اضافه شدن نیازمندی‌های عملکردی اولیه و جدول FR \\
\hline
3 & 19.۰۸.۱۴۰۴ & مریم رجبی & ویرایش بخش مقدمه، دامنه، کاربران و محیط عملیات \\
\hline
4 & 25.۰۸.۱۴۰۴ & مریم رجبی & اضافه شدن بخش نمودارها: Case Use و Diagram Activity و Diagram Class \\
\hline
5 & 29.۰۸.۱۴۰۴ & مریم رجبی & نسخه نهایی پیش از تحویل، شامل تمام بخش‌ها و اصلاحات \\
\hline
6 & 03.۰۸.۱۴۰۴ & مریم پورحسن & تعیین نیازمندی‌های پروژه – کاربران هدف – دامنه پروژه – ذی‌نفعان پروژه \\
\hline
7 & 11.۰۸.۱۴۰۴ & مریم پورحسن & تعیین شاخص‌های مدیریت – ریسک‌ها و چالش‌های بیزینسی – نقشه راه – نتیجه‌گیری \\
\hline
8 & 15.۰۸.۱۴۰۴ & مریم پورحسن & تعیین نیازمندی‌های غیر عملکردی \\
\hline
9 & 21.۰۸.۱۴۰۴ & مریم پورحسن & محدودیت‌ها و الزامات طراحی – معیارهای کیفیت و روش‌های اندازه‌گیری \\
\hline
10 & 25.۰۸.۱۴۰۴ & مریم پورحسن & ردیابی و تایید و پذیرش – روش‌های تایید و تست معیار پذیرش برای هر نیازمندی \\
\hline
11 & 30.۰۸.۱۴۰۴ & مریم پورحسن & اصلاح و بازبینی نهایی تمام بخش‌ها + نمودارهای Diagram Sequence و Diagram Flow Data \\
\hline
\end{tabular}
\vspace{0.5cm}

\begin{center}
\textbf{تاریخچه نسخه‌های سند SRS}
\end{center}

\vspace{0.5cm}
\vspace{0.5cm}

\end{document}
